\documentclass[12pt, a4paper, titlepage, headsepline]{scrartcl} %KOMA

\usepackage{graphicx} %Grafik
\usepackage{color} %Farbe
\usepackage[ngerman]{babel} %Deutsche Rechtschreibung
\usepackage[latin1]{inputenc} %Umlaute
\usepackage[bookmarksopen=true, pdfborder=false]{hyperref} %Lesezeichen

\pagestyle{headings} %Stil der Kopf- und Fu�zeilen festlegen

\setlength{\parindent}{0pt} %Erstzeileneinzug entfernen
\setlength{\parskip}{10pt} %Abs�tze durch Leerzeile trennen

\definecolor{uniblau}{cmyk}{1,0.83,0.35,0.24} %Farbe vom Unilogo
\addtokomafont{section}{\color{uniblau}} %�berschriften �ndern

%------------------------------------------------------------------------------------------------------------------

\begin{document}
\sffamily

\begin{titlepage} %Titelseite
	\noindent
	\begin{center}
		\includegraphics[width=0.75\textwidth]{uni-logo.png}\\
		\vspace{3cm}
		\LARGE �nderungsdokument\\
		\vspace{1cm}
		\huge \textbf{\textcolor{uniblau}{Software - Praktikum}}\\
		\vspace{1cm}
		\LARGE \textit{Sommersemester 2005}\\
		\LARGE \textit{Dr. Michael Tauber}\\
		\vspace{2cm}
		\LARGE \textit{Gruppe 1}\\
		\LARGE \textit{Betreuer Matthias Schnelte}\\
		\vspace{2cm}
		\large \today %aktuelles Datum
		\vfill
		\normalsize
	\end{center}

%------------------------------------------------------------------------------------------------------------------
	
\end{titlepage}

\rmfamily

\pagenumbering{Roman} %r�mische Nummerierung
\tableofcontents %Inhaltsverzeichnis
\listoffigures %Abbildungsverzeichnis
\listoftables %Tabellenverzeichnis
%\newpage
\pagenumbering{arabic} %arabische Nummerierung
\linespread{1.3} %Eineinhalb-facher Zeilenabstand
\normalsize

\section{�nderungen im Datenmodell}
Das Datenmodell wurde an einzelnen Stellen erg�nzt und teilweise ver�ndert. Neben einzelnen Objekt-Eigenschaften betrifft dies vor allem die nach dem Einreichen des Pflichtenhefts aufgekommene Anforderung, Textmodule als Grundlage verschiedener Objekte zu verwenden. Entsprechend wurde das Modell angepasst: Es existieren nun TextModule (Text und Audiodatei) und Listen von Textmodulen. Letztere finden Verwendung als Seiten oder Faktkarten. Durch diese Abstraktion ist ein einheitlicher Zugang zum Editieren von Texten erm�glicht worden.

\section{Persistenzstrategie}
Zus�tzlich zur geforderten Persistenz ist auch noch eine Implementierung einer speicherbasiertend Datenhaltung hinzugekommen. Diese ist vor allem m�glich um f�r Unit-Tests entsprechende Objekte zur Verf�gung zu stellen.

\section{Nicht umgesetzte Anforderungen}
Die Perspektivische Mauer kann nicht sinnvoll konfiguriert werden. Dazu m�sste der entsprechende Dialog stark �berarbeitet und eine Strategie festgelegt werden, wo die Konfiguration gespeichert wird. Speichert man die Konfiguration der Datenbank ist sie nur benutzerabh�ngig, kann jedoch nicht f�r unterschiedliche Rechner erfolgen. Speichert man sie auf dem Rechner, so kann f�r jeden Rechner eine individuelle Konfiguration erfolgen; die Einstellungen stehen jedoch an einem anderen Rechner nicht zur Verf�gung. Sinnvoll w�re eine hybride Strategie.

Die Wizards wurden nicht erhalten. Die Eingabe erfolgt fensterorientiert. Dies erscheint sinnvoll, da die Anwender des Systems sog. Power-User sind, w�hrend Wizards im Allgemeinen f�r Anwendungen empfohlen werden, die nur selten benutzt werden.

F�r die Festlegung der Ausfall-, Ausschluss- und Therapietage ist kein eigenes Widget implementiert worden. Der Benutzer kann nur Listen mit den entsprechenden Tagen verarbeiten. Auch der Periodenbeginn, Evaluationsbeginn und -ende k�nnen nur jeweils einzeln in einem Kalender selektiet werden. Es erfolgt keine �berpr�fung auf Sinnhaftigkeit der Daten (z.B. Ende vor Beginn). Der Honeymoon ergibt sich automatisch durch das Ende einer Evaluation und den Starttag der folgenden Periode.

Die einfache Installation der Vorg�ngegruppe wurde nicht erreicht, da kein Tool zur Verf�gung stand mit dem ein entsprechender Installer erzeugt werden kann.

Es wurden zwar alle Elemente in der GUI umgesetzt, jedoch ist diese Umsetzung teilweise rudiment�r, so dass empfohlen wird diese mit GUI-Experten zu �berarbeiten. Dabei k�nnen Struktur und Klassen �bernommen werden. Es ist jedoch ein Beautifing notwendig.

\section{Nicht vollst�ndig umgesetzte Anforderungen}
Die Equals-Methoden wurden nur in einigen Klassen ad hoc �berschrieben.

Das Auslagern der Beschriftung ist gro�teils erfolgt. Jedoch sind auch hier noch einige Klassen vorhanden, bei den Strings direkt im Quelltext kodiert sind.

\end{document}
